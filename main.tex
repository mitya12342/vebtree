\documentclass{article}

% Language setting
% Replace `english' with e.g. `spanish' to change the document language
\usepackage[russian]{babel}

% Set page size and margins
% Replace `letterpaper' with `a4paper' for UK/EU standard size
\usepackage[a4paper,top=2cm,bottom=2cm,left=3cm,right=3cm,marginparwidth=1.75cm]{geometry}

% Useful packages
\usepackage{amsmath}
\usepackage{graphicx}
\usepackage[colorlinks=true, allcolors=blue]{hyperref}

\title{Дерево ван Эмде Боаса}
\author{Голобородько Димитрий}

\begin{document}
\maketitle


\section{Введение}
\subsection{Суть и назначение}
Дерево ван Эмде Боаса — поисковая структура данных, представляющая собой дерево поиска, позволяющее хранить целые неотрицательные числа в интервале $[0;2^k-1)$ и осуществлять над ними операции find, insert, remove, successor, predecessor, min, max за $O(log{k})$ при затратах памяти $\Theta(2^k)$.
\subsection{История}
Структура разработана осенью 1974 года Питером ван Эмде Боасом во время его трёхмесячного пост-докторского резиденства в Корнеллском университете и представлена в 1975 году. Автор структуры отмечает, что на его ранние публикации оказывал влияние культурный климат в алгоритмике середины 1970-х годов, на фоне которого структура была разработана для машины указателей (pointer machine). Причина в том, что рекурсивный подход требует адресных вычислений. Данные операции не допускались в модели машины с произвольным доступом (RAM), которая была стандартной моделью в развивающейся области исследований проектирования и анализа алгоритмов в 1974 году. Недостатком этого подхода является то, что он приводит к довольно сложным алгоритмам, которые и сегодня трудно корректно реализовать. 
\subsection{Применение}
В сети встречаются упоминания применения дерева ван Эмде Боаса в алгоритмах на графах и вычислительной геометрии, а также в сетевых маршрутизаторах.
\section{Описание}
\subsection{Структура данных}
Дерево ван Эмде Боаса является рекурсивной структурой, каждый узел которого является корнем дерева и содержит в себе:
\begin{itemize}
    \item u - размерность дерева, бит
    \item min, max - хранят минимальный и максимальный элемент дерева
    \item summary - указатель на справочную структуру, которая также является деревом ван Эмде Боаса с размерностью $\sqrt{u}$ бит
    \item cluster - массив из $\sqrt{u}$ указателей на поддеревья с размерностью $\sqrt{u}$
\end{itemize}
\section{Формальная постановка задачи}
Исследовать и реализовать дерево ван Эмде Боаса...
\section{Реализация}
\subsection{Тесты}
\subsection{Производительность}
\section{Заключение}


\bibliographystyle{alpha}
\bibliography{sample}

\end{document}
